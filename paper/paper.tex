\documentclass[fleqn,10pt]{wlscirep}
\title{Improved Predictions Using Model Uncertainty}

\author[2]{John McGowan}
\author[1,*]{Yair Daon}
\author[2]{Hao Ye}
\author[2]{Melissa Carter}
\author[2]{George Sugihara}

\affil[1]{NYU, Courant Institute of Mathematical Sciences, New York, NY 10012 USA}
\affil[2]{University of California San Diego, Scripps Institution of Oceanography, La Jolla, CA 92037, USA}

\affil[*]{yair.daon@gmail.com}

\affil[+]{these authors contributed equally to this work}

%\keywords{Keyword1, Keyword2, Keyword3}

\begin{abstract}
  Any method that hopes to make predictions should also be aware of
  its limitations --- i.e. how much it is uncertain of its own ability
  to make accurate predictions. We use an established method (S-maps)
  for nonlinear prediction and define a measure of its prediction
  uncertainty. We use this measure of uncertainty to rank and average
  different sets of variables used to make predictios of abundance of
  Chlorophyll near the coasts of southern California. We observe good
  prediction skill in most days and note that when most sets of
  variables have high prediction uncertainty, extreme events occur.
\end{abstract}

\newcommand{\un}{\text{Var}}
\newcommand{\x}{\mathbf{x}}
\newcommand{\pred}{\mathbf{P}}

\begin{document}

\flushbottom
\maketitle

\thispagestyle{empty}

\section{Introduction}
Nonlinear prediciton using S-maps \cite{Smaps} and simplex projection
\cite{simplex} are by now an establised set of techniques used for
analyzing natural systems \cite{Salmon,Influenza,Neurons}. The idea is
simple: first, reconstruct an attractor manifold for (an assumed)
underlying dynamical system, using various (potentially time-lagged)
observables, both biotic and a-biotic. We denote the system's position
on the attractor (i.e. its state) by $\omega$. A prediction of some
observable at some state $\omega$ is made by interpolating the value of
said observable using neighbouring states on the reconstructed
manifold, either by averaging (simplex projection) or by employing a
\emph{local} linear model (S-maps).

Choosing different observables to reconstruct the attractor (represent
the state) amounts to a coordinate change. Such coordinate change is a
deformation of the manifold which may involve stretching, skewing,
bending, and reflecting in the ambient space. We characterize the
quality of a specific choice of coordinates as the prediction
uncertainty, defined by the divergence of trajectories on the
representation of the reconstructed manifold. We then use this measure
of uncertainty to choose the best representations for each state to
make predictions. Details may be found in section
\ref{section:methods}.

Using the uncertainty measure outlined in \ref{section:methods} we
approach the problem of predicting Chlorophyll abundance near the
coasts of southern California. For a thorough introduction, see
\cite{AlgaeBloom}. There is was found that six environmental
observables are causally related to the dynamics of Chlorophyll near
the coasts of southern California: Silicate, Nitrate, Nitrite, wind
speed, water density and water temperature, though the exact
specifications of these is irrelevant. following \cite{AlgaeBloom},
predictions are made using different sets of four potentially
time-lagged observables (different representations), of which
Chlorophyll abundance is always a member. For example, a
representation may be Chlorophyll(0), Silicate(0), Silicate(-1) and
density(0), where the number in parentheses represents the time lag of
the corresponding observable in weeks. We take lags of one or two
weeks so we have $6 \cdot 3 + 2 = 20$ observables to choose from ---
lags of 0,1, or 2 weeks for the six environmental observables plus lag
of 1 or 2 for weeks Chlorophyll. This amounts to a total of
$\binom{20}{3} = 1140$ sets of observables (representations) to choose
from.

For each prediction time, we average a fraction of the least uncertain
(most confident) predictors, each uses a different representation to
reconstruct the attractor manifold. We explore two different averaging
schemes and compare their performance. The ultimate goal is predicting
algae blooms --- events where Chlorophyll levels are in the top 5\% of
all Chlorophyll observations.

\section{Results}\label{section:results}
Performance on bloom days is not strong, but performance of either
weighted predictor on non-bloom days gives $\rho = 0.54$ when
calculated using leave-one-out cross validation over the entire data.

\begin{figure}
  \centering
  \includegraphics[width=0.45\textwidth]{../plots/exp_time_series_oos.pdf}
  \includegraphics[width=0.45\textwidth]{../plots/exp_weighted_predictions_oos.pdf}
  \caption{Weighting predictors by $\exp \left ( -\un (\x) \right
    )$. Left \textbf{(a)} observed and predicted Chlorophyll
    abundance. Right \textbf{(b)} prediction skill for fraction of most
    confident models averaged.}
  \label{fig:exp}
\end{figure}

\begin{figure}
\centering
  \includegraphics[width=0.45\textwidth]{../plots/precision_time_series_oos.pdf}
  \includegraphics[width=0.45\textwidth]{../plots/precision_weighted_predictions_oos.pdf}
  \caption{Weighting predictors by $1 \big / \un (\x)$. Left
    \textbf{(a)} observed and predicted Chlorophyll abundance. Right
    \textbf{(b)} prediction skill for fraction of most confident models
    averaged.}
  \label{fig:prec}
\end{figure}

It is interesting to note that the mean of the prediction uncertainties of
all $n:=\binom{20}{3}$ models is strongly correlated to Chlorophyll
levels ($\rho = 0.7$), as can be seen in figure \ref{fig:var chl}.

\begin{figure}[ht]
\centering
\includegraphics[width=0.5\textwidth]{../plots/var_chl_oos.pdf}
\caption{Perhaps surprisingly, uncertainty seems to be well correlated
  with the abundance of Chlorophyll ($\rho = 0.7$).}
\label{fig:var chl}
\end{figure}

Another interesting phenomenon is that the two weighting schemes give
rise to almost identical predictions in \ref{fig:exp}.\textbf{a} and
\ref{fig:prec}.\textbf{a} when using the fraction of most confident
representations maximizing \ref{fig:exp}.\textbf{b} and
\ref{fig:prec}.\textbf{b}, respectively.

\section{Discussion}
We showed there is a potential gain to be had by using different
representations in S-maps according to the level of uncertainty each
representation has in each time. Considering the Chlorophyll abundance
in the pacific ocean near southern California, prediciton skill is
high in days where algae blooms do not occur. This goes to show that
there is some predictability in the phytoplanktons abundance but also
that a more detailed study is required to understand algae blooms
(e.g. study by species abundance, as opposed to Chlorophyll
abundance).

\section{Methods}\label{section:methods}
For an $E$-dimensional attractor, denote a state on the attractor
$\omega^{(t)}$. This state is abstract and never observed or known. The
only access we have to $\omega^{(t)}$ via different $E$-dimensional
representations $\x^{(t)} = \x(\omega^{(t)}) =
(x_1^{(t)},...,x_E^{(t)})$. For example, we may take time lags of the
same time series, so $\x^{(t)} = (x^{(t)},...,x^{(t-E+1)})$. For every
$t$, we are also given an observation $y^{(t)}$. Note that in many
cases, we may want to predict the time evolution of (say) the first
coordinate and then $y^{(t)} := x_1^{(t-1)}$. The S-map prediction of
$y$ given a representation $\x = \x(\omega)$ (we drop $\omega$ for
convenience) and a nonlinearity parameter $\theta$ is found by first
solving the (local) least squares problem for $\beta$:
\begin{equation}
  \beta = \beta(\x) := \arg \min_{\hat{\beta}} \|W(X \hat{\beta} - \mathbf{y}) \|_2^2,
\end{equation}
with
\begin{equation*}
  X =
  \begin{bmatrix}
    \rule[.5ex]{3.5em}{0.4pt} & \x^{(1)} & \rule[.5ex]{3.5em}{0.4pt} \\
                              &  \vdots         &                 \\
    \rule[.5ex]{3.5em}{0.4pt} & \x^{(T)} & \rule[.5ex]{3.5em}{0.4pt} \\
  \end{bmatrix},
  \mathbf{y} =
  \begin{bmatrix}
    y^{(1)} \\
    \vdots \\
    y^{(T)}
  \end{bmatrix},
  W := diag( w_1,...,w_T)  \text{ and } w_t: =\exp( -\theta \| \x - \x^{(t)}\| ).
\end{equation*}
Having calculated $\beta$, we predict $y = \beta' \x = \sum_{i=1}^E \beta_i
x_i$. The uncertainty measure we intruduce is defined as
\begin{equation}\label{eq:uncertainty}
  \un (\x) := \sum_{t=1}^T w_t (y - y^{(t)})^2
  \bigg / \sum_{t=1}^T w_t = \sum_{t=1}^T w_t (\beta' \x -
  y^{(t)})^2 \bigg / \sum_{t=1}^T w_t.
\end{equation}
This is in contrast to the similar \emph{fitting error} that we found
insufficient for our purposes
\begin{equation*}
\text{Fitting Error}(\x) := \sum_{t=1}^T w_t ( \beta'
\x^{(t)} - y^{(t)})^2 \bigg / \sum_{t=1}^T w_t.
\end{equation*}
The former represents the divergence in trajectories of the time
series, whereas the latter merely represents a fitting error, hence
underestimating the prediction uncertainty associated with
$\x$.

Assume we consider a fixed (unknown) state $\omega$ and are given a
set of representations $\x_i, i =1,...,n$ (e.g. $n = \binom{20}{3}$ in
our case) with associated predictions $y_i$ and uncertainties $\un_i =
\un(\x_i)$ such that $\un_i$ are increasing. We can choose a quantile
$0 < q \leq 1$ and average these predictors with uncertainty in the
$q$th quantile for prediction, so we use only the $q$ fraction of most
confident predictors and we can use the uncertainties to define
averaging weights. For example, we can take the exponentially-weighted
predictor
\begin{equation*}
  y_{\text{exp}}  := \sum_{i=1}^{nq} \exp(-\un_i) y_i(\x) \bigg / \sum_{i=1}^{nq} \exp(-\un_i).
\end{equation*}
This weighting scheme was used to generate figure \ref{fig:exp}. We
may also weight by the inverse of the uncertainty, resulting in the
following weighting scheme:
\begin{equation*}
  y_{\text{precision}} := \sum_{i=1}^{nq} \frac{1}{\un_i} y_i \bigg / \sum_{i=1}^{nq} \frac{1}{\un_i}.
\end{equation*}
This weighting scheme was used to generate figure \ref{fig:prec}.

The datasets analysed during the current study are not publicly
available due to them being proprietary but are available from the
corresponding author on reasonable request. The results in section
\ref{section:results} can be generated by anyone who has the original
data. Documented code can be downloaded from
\url{http://github.com/yairdaon/algae}. Given the data, one should
type \emph{make everything} in the main directory to generate all the
plots and printouts. The code requires the package rEDM version 0.6.0
or higher.

\bibliography{bibi}

\section*{Acknowledgements}
This research was supported in part by an appointment with the NSF
Mathematical Sciences Summer Internship Program sponsored by the
National Science Foundation, Division of Mathematical Sciences
(DMS). This program is administered by the Oak Ridge Institute for
Science and Education (ORISE) through an interagency agreement between
the U.S. Department of Energy (DOE) and NSF. ORISE is managed by ORAU
under DOE contract number DE-SC0014664.

\section*{Author contributions statement}
YD prepared the manuscript, figures and code. YD, GS and HY
suggested the uncertainty measure. JM and MC collected the data.

\section*{Additional information}

To include, in this order: \textbf{Accession codes} (where
applicable); The authors declare no competing financial interest.

The corresponding author is responsible for submitting a
\href{http://www.nature.com/srep/policies/index.html#competing}{competing
  financial interests statement} on behalf of all authors of the
paper. This statement must be included in the submitted article file.

\end{document}
