\documentclass[fleqn,10pt]{wlscirep}
\title{Improved Predictions Using Model Uncertainty}

\author[2]{John McGowan}
\author[1,*]{Yair Daon}
\author[2]{Hao Ye}
\author[2]{Melissa Carter}
\author[2]{George Sugihara}

\affil[1]{NYU, Courant Institute of Mathematical Sciences, New York, NY 10012 USA}
\affil[2]{University of California San Diego, Scripps Institution of Oceanography, La Jolla, CA 92037, USA}

\affil[*]{yair.daon@gmail.com}

\affil[+]{these authors contributed equally to this work}

%\keywords{Keyword1, Keyword2, Keyword3}

\begin{abstract}
  Any method that hopes to make predictions should also be aware of
  its limitations --- i.e. how much it is uncertain of its own ability
  to make accurate predictions. We use the same method of predictios
  but with different variables to make predictions of abundance of
  Chlorophyll in the pacific ocean by southern California. We suggest
  a novel method for estimating prediction uncertainty and use it to
  rank and weight the predictors. We see good prediction skill in the
  case where algae bloom events do not occur.
\end{abstract}

\newcommand{\un}{\text{Uncertainty}}
\newcommand{\x}{\mathbf{x}}
\newcommand{\pred}{\mathbf{P}}

\begin{document}

\flushbottom
\maketitle

\thispagestyle{empty}

\section{Introduction}
Nonlinear prediciton using S-maps \cite{Smaps} and simplex projection
\cite{simplex} are by now an establised set of techniques used for
analyzing natural systems \cite{Salmon,Influenza,Neurons}. The basic
idea is simple: first, reconstruct an attractor manifold for the
underlying dynamical system, using various (potentially time-lagged)
variables, both biotic and a-biotic. A prediction of some variable at
some state is made by interpolating the value of said variable using
neighbouring points on the reconstructed manifold, either by averaging
(simplex projection) or by employing a \emph{local} linear model
(S-maps).

Choosing different variables to reconstruct the attractor amounts to a
coordinate change --- a deformation of the manifold which may involve
stretching, skewing, bending, and reflecting of the ambient space. We
characterize the quality of a specific choice of coordinates as the
prediction uncertainty dictated by divergence of trajectories on the
reconstructed manifold. We then use this measure of uncertainty to
choose the best models in every state to make predictions. Details may
be found in section \ref{section:methods}.

Using the uncertainty measure described in \eqref{eq:uncertainty} we
approach the problem of predicting Chlorophyll abundance in the seas
of southern California. For a thorough introduction, see
\cite{AlgaeBloom}.

\section{Results}\label{section:results}
 It was found in \cite{AlgaeBloom} that six variables are causally
 related to the dynamics of Chlorophyll in the seas of southern
 California --- Silicate, Nitrate, Nitrite, wind speed, water density
 and water temperature --- though the exact specifications of these is
 irrelevant. We make predictions using different sets (``models'') of
 four (as shown in \ref{AlgaeBloom}) potentially time-lagged
 variables, of which Chlorophyll abundance is always a member. An
 example set of variables may be Chlorophyll(0), Silicate(0),
 Silicate(-1) and density(0). The number in parentheses represents the
 time lag of the corresponding variable. We take lags of one or two
 weeks so we have $6 \cdot 3 + 2 = 20$ variables to choose --- lags of
 0,1, or 2 weeks for the six environmental variables plus lag of 1 or
 2 for weeks Chlorophyll. This amounts to $n := \binom{20}{3} = 1140$
 sets of variables (``models'') to choose from. For each prediction time, we
 average a fraction of the least uncertain (most confident) models.

\begin{figure}[ht]
\centering
\includegraphics[width=0.4\linewidth]{../plots/exp_time_series_oos.pdf}
\includegraphics[width=0.4\linewidth]{../plots/exp_weighted_predictions_oos.pdf}
\caption{Weighting predictors by $\exp \left ( -\un (\x) \right )$}
\label{fig:exp}
\end{figure}

\begin{figure}[ht]
  \centering
\includegraphics[width=0.4\linewidth]{../plots/precision_time_series_oos.pdf}
\includegraphics[width=0.4\linewidth]{../plots/precision_weighted_predictions_oos.pdf}
\caption{Weighting predictors by $1 / \un (\x)$}
\label{fig:prec}
\end{figure}

It is also interesting to note that the sum of prediction uncertainty
of all $n:=\binom{20}{3}$ models is strongly correlated to Chlorophyll levels, as
can be seen in figure \ref{fig:var chl}.

\begin{figure}[ht]
\centering
\includegraphics[width=0.5\linewidth]{../plots/var_chl_oos.pdf}
\caption{Perhaps surprisingly, uncertainty seems to be well correlated
  with the abundance of Chlorophyll ($\rho = 0.7$).}
\label{fig:var chl}
\end{figure}

\section{Discussion}
We show there is much to gain by using different variables in S-maps
according to the level of uncertainty each model has. Considering the
Chlorophyll abundance in the pacific ocean near southern California,
prediciton skill is high in days where algae blooms do not occur. This
goes to show that there is a lot of predictability in the
phytoplanktons abundance but also that a more detailed study is
required to understand algae blooms (e.g. study by species abundance,
as opposed to Chlorophyll abundance).

\section{Methods}\label{section:methods}
We introduce a measure of prediction uncertainty for the S-map. We are
given one or several time series which we use to reconstruct an $E$
dimensional attractor. Denote the state on the attractor $\x^{(t)} =
(x_1^{(t)},...,x_E^{(t)})$ with $t=1,...,T$. In the univariate case we
take time lags of the same time series, so $\x^{(t)} =
(x^{(t)},...,x^{(t-E+1)})$. For every $t$, we are also given an
observation $y^{(t)}$. Note that in many cases, we may want to predict
the time evolution of (say) the first coordinate and then $y^{(t)} :=
x_1^{(t-1)}$. The S-map prediction for a new $\x$ with a given
nonlinearity parameter $\theta$ is found by first solving the (local)
least squares problem for $\beta$:
\begin{equation}
  \beta = \beta(\x) := \arg \min_{\hat{\beta}} \|W(X \hat{\beta} - \mathbf{y}) \|_2^2,
\end{equation}
with
\begin{equation*}
  X =
  \begin{bmatrix}
    \rule[.5ex]{3.5em}{0.4pt} & \x^{(1)} & \rule[.5ex]{3.5em}{0.4pt} \\
                              &  \vdots         &                 \\
    \rule[.5ex]{3.5em}{0.4pt} & \x^{(T)} & \rule[.5ex]{3.5em}{0.4pt} \\
  \end{bmatrix},
  \mathbf{y} =
  \begin{bmatrix}
    y^{(1)} \\
    \vdots \\
    y^{(T)}
  \end{bmatrix},
  W := diag( w_1,...,w_T)  \text{ and } w_t: =\exp( -\theta \| \x - \x^{(t)}\| ).
\end{equation*}
Having calculated $\beta$, we predict $y = \beta' \x = \sum_{i=1}^E \beta_i
x_i$. The novel uncertainty measure we intruduce is defined as
\begin{equation}\label{eq:uncertainty}
  \un (\x) := \sum_{t=1}^T w_t (y - y^{(t)})^2
  \bigg / \sum_{t=1}^T w_t = \sum_{t=1}^T w_t (\beta' \x -
  y^{(t)})^2 \bigg / \sum_{t=1}^T w_t.
\end{equation}
This is in contrast to the similar \emph{fitting error} that we found
insufficient
\begin{equation*}
\text{Fitting Error}(\x) := \sum_{t=1}^T w_t ( \beta'
\x^{(t)} - y^{(t)})^2 \bigg / \sum_{t=1}^T w_t.
\end{equation*}
The former represents the divergence in trajectories of the time
series, whereas the latter merely represents a fitting error, hence
underestimating the prediction uncertainty associated with
$\x$.

Assume we consider a fixed $\x$ and are given a set of predictors
$\pred_i, i =1,...,n$ with associated uncertainties $\un_i(\x)$ and
assume these are increasing (so better predictors have smaller
indices). We can choose a quantile $0 < q \leq 1$ and average these
predictors with uncertainty in the $q$th quantile for prediction, so
we use only the $q$ fraction of most confident predictors. We can use
the uncertainties as weights. For example, we can take the
exponentially-weighted predictor
\begin{equation*}
  \pred_{\text{Exp}} (\x ) := \sum_{i=1}^{nq} \exp(-\un_i(\x)) \pred_i(\x) / \sum_{i=1}^{nq} \exp(-\un_i(\x)).
\end{equation*}
 This weighting scheme was used to generate figure \ref{fig:exp}. We
 may also weight by the inverse of the uncertainty, resulting in the
 following weighting scheme:
\begin{equation*}
  \pred_{\text{Precision}} (\x ) := \sum_{i=1}^{nq} \frac{1}{\un_i(\x)} \pred_i(\x) / \sum_{i=1}^{nq} \frac{1}{\un_i(\x)}.
\end{equation*}
This weighting scheme was used to generate figure \ref{fig:prec}.

It is crucial to note that $\x$ represent an abstract state on the
attractor manifold. In this respect, a predictor $\pred_i$ also
involves a specific set of coordinates used for prediction.

The results in section \ref{section:results} can be generated by
anyone who has the original data, which is, unfortunately,
proprietary. Documented code can be downloaded from
\url{http://github.com/yairdaon/algae}. Given the data, one should
type \emph{make everything} in the main directory to generate all the
plots. This requires the package rEDM version 0.6.0 or higher.

\bibliography{bibi}

\section*{Acknowledgements (not compulsory)}
This research was supported in part by an appointment with the NSF
Mathematical Sciences Summer Internship Program sponsored by the
National Science Foundation, Division of Mathematical Sciences
(DMS). This program is administered by the Oak Ridge Institute for
Science and Education (ORISE) through an interagency agreement between
the U.S. Department of Energy (DOE) and NSF. ORISE is managed by ORAU
under DOE contract number DE-SC0014664.

\section*{Author contributions statement}
Must include all authors, identified by initials, for example:
A.A. conceived the experiment(s), A.A. and B.A. conducted the
experiment(s), C.A. and D.A. analysed the results. All authors
reviewed the manuscript.

\section*{Additional information}

To include, in this order: \textbf{Accession codes} (where
applicable); \textbf{Competing financial interests} (mandatory
statement).

The corresponding author is responsible for submitting a
\href{http://www.nature.com/srep/policies/index.html#competing}{competing
  financial interests statement} on behalf of all authors of the
paper. This statement must be included in the submitted article file.

\end{document}
